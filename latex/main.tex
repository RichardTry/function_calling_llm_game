\documentclass[14pt]{extarticle}
\usepackage[left=3.5cm, right=1.5cm, vmargin=2cm]{geometry}
\usepackage[utf8]{inputenc}
\usepackage{amsmath}
\usepackage{amssymb}
\linespread{1.3}
\usepackage{siunitx}

\usepackage{caption}
\usepackage{subcaption}
\renewcommand\thesubfigure{\asbuk{subfigure}}


\usepackage{enumitem}

\usepackage{float}
\usepackage{cases}
\usepackage[T2A]{fontenc}
\usepackage[russian]{babel}   %% загружает пакет многоязыковой вёрстки
\usepackage{graphicx}% Include figure files
\usepackage{dcolumn}% Align table columns on decimal point
\usepackage{bm}% bold math
\usepackage{hyperref}% add hypertext capabilities
\usepackage{xcolor}
\usepackage[mathlines]{lineno}% Enable numbering of text and display math
%\linenumbers\relax % Commence numbering lines

\bibliographystyle{IEEEtran} 
\setlength{\emergencystretch}{30pt}
\usepackage{hyphenat}
\hyphenation{ма-те-ма-ти-ка вос-ста-нав-ли-вать}

\definecolor{linkcolor}{HTML}{0000D5} % цвет ссылок
\definecolor{urlcolor}{HTML}{0000D5} % цвет гиперссылок
\definecolor{citecolor}{HTML}{0000D5}

 
\hypersetup{pdfstartview=FitH,  linkcolor=linkcolor,urlcolor=urlcolor, colorlinks=true}
 
%\usepackage[showframe,%Uncomment any one of the following lines to test 
%%scale=0.7, marginratio={1:1, 2:3}, ignoreall,% default settings
%%text={7in,10in},centering,
%%margin=1.5in,
%%total={6.5in,8.75in}, top=1.2in, left=0.9in, includefoot,
%%height=10in,a5paper,hmargin={3cm,0.8in},
%]{geometry}

\usepackage{etoolbox}
\makeatletter
% \frontmatter@RRAP@format is responsible for the parentheses
\patchcmd{\frontmatter@RRAP@format}{(}{}{}{}
\patchcmd{\frontmatter@RRAP@format}{)}{}{}{}

\makeatother

\begin{document}
	\begin{titlepage}
        \newgeometry{left=1.5cm, right=1.5cm, vmargin=2cm}
		\begin{center}
			{\normalsize{ФЕДЕРАЛЬНОЕ ГОСУДАРСТВЕННОЕ АВТОНОМНОЕ ОБРАЗОВАТЕЛЬНОЕ УЧРЕЖДЕНИЕ ВЫСШЕГО ОБРАЗОВАНИЯ}}\\
                \vspace{0.1cm}
			{\normalsize{«НАЦИОНАЛЬНЫЙ ИССЛЕДОВАТЕЛЬСКИЙ УНИВЕРСИТЕТ \\ “ВЫСШАЯ ШКОЛА ЭКОНОМИКИ”»}}\\
			\vspace{1cm}
			{\normalsize{Факультет Компьютерных Наук}}\\
                \vspace{1.5cm}
                

                {\normalsize{Базров Степан Артурович}}\\
                \vspace{0.15cm}
			{\normalsize{\textbf{Создание игры при помощи LLM с Function Calling}}}\\ 
                \vspace{0.15cm}
                {\normalsize{Выпускная квалификационная работа}}\\
                {\normalsize{по направлению подготовки НОМЕР И НАЗВАНИЕ}}\\
                {\normalsize{образовательная программа «Машинное обучение и высоконагруженные системы»}} \\
            \end{center}
        \vspace{4.2cm}
        \begin{minipage}{0.4\textwidth}
        \hfill
	\end{minipage}
	\hspace{2cm}
	\begin{minipage}{0.4\textwidth}
		\begin{flushleft}
			{\normalsize{{Научный руководитель}}}\\
                {\normalsize{должность}}\\
                {\normalsize{ФИО}} \\
		\end{flushleft}
	\end{minipage}
            \vspace{4.0cm}
        \begin{center}
            Москва 2025
        \end{center}
        \restoregeometry
	\end{titlepage}
\tableofcontents
\newpage
\section*{Аннотация}
С развитием больших языковых моделей (LLM) разработчики компьютерных игр начали активно экспериментировать с их интеграцией в игровой процесс. Одним из первых и наиболее известных примеров стал AI Dungeon, в котором LLM используется почти напрямую в виде текстового чата с игроком. Такой подход привел к тому, что вместо сюжетной игры с четкой игровой логикой, пользователи зачастую получают бесконечный текстовый диалог, не ограниченный темой или контекстом. Модель может легко отвлечься на совершенно случайные разговоры, выступать в роли психолога, советника по личным вопросам или давать ответы, не относящиеся к игровому процессу.

Для преодоления данных ограничений требовалось создать игру, обладающую типичными признаками видеоигры: с визуализацией игрового мира, персонажей и четко определенным состоянием игры (например, локацией и инвентарём персонажа). Чтобы избежать проблем забывания контекста, было решено использовать подход function calling, при котором внутреннее состояние игры хранится отдельно, подобно конвенциональным видеоиграм. В этом случае LLM служит интерфейсом взаимодействия, преобразующим текстовые запросы пользователя в вызовы строго заданных функций API игрового движка.

В рамках работы был разработан прототип ролевой игры с использованием fine-tuned языковой модели gemma-2-9b-it-russian-function-calling-GGUF и подхода function calling. Была реализована архитектура, позволяющая преобразовывать произвольные текстовые запросы игрока в конкретные вызовы функций API самописного игрового движка на основе библиотеки pygame. Такой подход позволил обеспечить управляемость и целостность игрового состояния, сохранив при этом все преимущества интерактивности и гибкости, которые предоставляет интеграция LLM в игровой процесс.
\section*{Ключевые слова}
Large Language Model, Function Calling, Видеоигры, Ролевая игра, Pygame
\newpage
% \newpage
\section{Введение}
С момента зарождения ролевых игр разработчики стремились к максимальному оживлению игрового мира и персонажей (NPC). Создавались различные подходы к реализации диалоговых систем, включая попытки внедрения свободных, нелинейных диалогов. Одним из примеров может служить система диалогов в игре Morrowind, где игроку предоставлялся по сути словарь доступных тем для обсуждения, что создавало ощущение более живого и гибкого общения. Помимо этого, разработчики стремились предоставить игрокам максимальную свободу действий, приближая игровой опыт к формату настольных ролевых игр, таких как Dungeons \& Dragons, где игрок может напрямую сообщить ведущему любое действие без необходимости выбирать его из ограниченного списка. Появление LLM породило у разработчиков надежду на решение многолетней проблемы создания по-настоящему живых, осмысленных NPC, способных поддерживать диалог практически на любые темы и динамически реагировать на действия игрока.

Современные достижения в области больших языковых моделей (LLM, Large Language Models) открывают значительные перспективы для разработки интерактивных приложений и компьютерных игр. Эти модели, такие как GPT, LLaMA и их производные, продемонстрировали способность генерировать связные, осмысленные тексты, эффективно обрабатывать запросы пользователей и имитировать живое человеческое общение. В связи с этим разработчики игр проявляют повышенный интерес к интеграции подобных решений в игровые системы, языковые модели стали восприниматься как потенциальное универсальное средство, способное радикально преобразить взаимодействие с игровым миром.

Одним из наиболее известных примеров применения LLM в игровой сфере стала игра AI Dungeon. Несмотря на новаторский подход и популярность среди пользователей, AI Dungeon представляет собой преимущественно текстовый квест, в котором языковая модель взаимодействует с пользователем в режиме чата с минимальными ограничениями и практически без дополнительного контроля игрового контекста. Это ведет к тому, что игровой процесс зачастую становится несфокусированным: пользователи свободно отклоняются от сюжетной линии, а модель легко переключается на темы, не имеющие отношения к игре, выполняя роль психолога, советника или просто собеседника на случайные темы.

Подобные ограничения существенно снижают потенциал применения LLM в более традиционных компьютерных играх с четко заданным состоянием и визуальным представлением. Игры, реализованные с помощью LLM «в лоб», лишены четкой игровой структуры и не способны надежно поддерживать важные параметры, такие как положение персонажа на карте, состояние инвентаря или текущие задания.

Таким образом, возникает задача создания архитектуры интеграции языковых моделей в игры, которая позволяет сохранять все преимущества гибкости и реалистичности взаимодействия с LLM, но при этом обеспечивает строгий контроль и четкое управление игровым состоянием. Для этого в данной работе предлагается использовать подход function calling, предполагающий вынесение игрового состояния и логики в отдельную структуру, управляемую конкретными функциями, которые языковая модель может вызывать по мере необходимости. В этом случае модель служит посредником между игроком и игровым движком, преобразуя текстовые команды пользователя в формализованные вызовы API игры.

Целью данной работы является разработка и экспериментальная проверка подобной архитектуры на примере ролевой игры с самописным движком на базе библиотеки pygame и специально дообученной языковой моделью gemma-2-9b-it-russian-function-calling-GGUF. В рамках работы разработана модульная система, состоящая из игровой логики, системы хранения состояния игрового мира на основе SQLite, системы генерации карт и квестов через YAML-описания, а также графа взаимоотношений NPC для динамической генерации заданий.

Проведенные эксперименты показали, что предложенный подход обеспечивает высокую точность и скорость реакции системы на пользовательские запросы, при этом сохраняя полную управляемость игрового процесса и предотвращая отклонения модели от заданного контекста игры.

Данная работа структурирована следующим образом: во второй главе проводится обзор существующих решений в области использования LLM в играх и технологии function calling. В третьей главе формализуется постановка задачи и описываются требования к разрабатываемой системе. В четвертой главе подробно описывается архитектура предлагаемой системы, включая используемые технологии и компоненты. В пятой главе представлена детальная информация о реализации прототипа игры. В шестой главе описаны методики и результаты экспериментального исследования. Седьмая глава посвящена обсуждению полученных результатов и выводам, а также намечены перспективы дальнейших исследований и применения разработанного подхода.

\section{Обзор существующих решений}
\subsection{Использование LLM в компьютерных играх}
\subsection{Проблемы текущих подходов (пример AI Dungeon и др.)}
\subsection{Механизм function calling и его применение в системах с LLM}
\subsection{Выводы по обзору существующих решений}

\section{Постановка задачи}
\subsection{Формулировка задачи и требований к разрабатываемой системе}
\subsection{Критерии успешности реализации прототипа}

\section{Архитектура разработанной системы}
\subsection{Общая схема взаимодействия компонентов}
\subsection{Используемая языковая модель gemma-2-9b-it-russian-function-calling-GGUF}
\subsection{Самописный игровой движок на базе pygame}
\subsection{Интерфейс взаимодействия (описание API и вызываемых функций)}

\newpage
\section{Реализация прототипа игры}
\subsection{Структура и логика игрового движка}
Для реализации данной работы был разработан игровой движок, построенный на базе библиотеки \texttt{Pygame}. Движок минималистичен и расширяем с точки зрения контента, о чём будет описано ниже. В задачи движка входит поддержка состояния игрового мира, обработка ввода пользователя и визуализация текущего состояния, включая карту и действия игрока.

\subsubsection*{Основные компоненты движка}

\begin{enumerate}
    \item \textbf{Игровое поле и визуализация карты}

    Игровой мир представлен в виде двумерного массива тайлов. Тайлы определяются YAML-схемами, хранящимися в папке \texttt{tiles}, и подгружаются через класс \texttt{Content}. Для каждого типа тайла предусмотрено отдельное изображение, которое масштабируется под размер клетки (\texttt{TILE\_SIZE}) и используется для отрисовки карты.

    Генерация карты реализована в классе \texttt{World} и осуществляется с помощью шума Перлина (Perlin Noise). В зависимости от высотных значений выбирается тип поверхности (вода, песок, земля, трава). Таким образом обеспечивается процедурная генерация карты, визуально разнообразной и пригодной для навигации.

    \item \textbf{Игрок}

    Объект игрока представлен классом \texttt{Player}, который содержит координаты, изображение, а также логику перемещения. Игрок перемещается по клеткам с помощью клавиш WASD или стрелок. Чтобы избежать слишком быстрого перемещения, введена система задержек (\texttt{move\_delay}), ограничивающая частоту перемещений. Позиция игрока сохраняется с плавающей точностью, что позволяет в будущем реализовать более плавное движение.

    \item \textbf{Интерфейс пользователя}

    Элементы интерфейса в игре реализованы с помощью вспомогательной библиотеки \texttt{pygame\_gui} и в основном представлены текстовым полем для ввода команд (промптов) игроком. Это поле обрабатывает текстовые команды, которые затем передаются в языковую модель для интерпретации и трансформации в вызовы игровых функций (function calling).

    \item \textbf{LLM}

    Элементы интерфейса в игре реализованы с помощью вспомогательной библиотеки \texttt{pygame\_gui} и в основном представлены текстовым полем для ввода команд (промптов) игроком. Это поле обрабатывает текстовые команды, которые затем передаются в языковую модель для интерпретации и трансформации в вызовы игровых функций (function calling).

    \item \textbf{Рендеринг}

    В каждом кадре происходит отрисовка карты, интерфейса и игрока. Метод \texttt{draw()} у каждого компонента отвечает за его визуализацию. При этом игровая карта отрисовывается через объект \texttt{World}, проходя по всем координатам и вызывая отрисовку соответствующего тайла. Игрок рендерится поверх карты.

    \item \textbf{Игровой цикл}

    Центральной частью движка является основной цикл, в котором обрабатываются события (ввод с клавиатуры и из текстового поля), обновляется состояние игрока и интерфейса, и выполняется перерисовка окна. Частота обновления ограничена 60 кадрами в секунду.
\end{enumerate}

Таким образом, игровой движок обеспечивает базовую структуру, необходимую для функционирования текстовой ролевой игры с визуальной составляющей и возможностью взаимодействия через LLM. Его модульная архитектура позволяет в дальнейшем расширять механику игры, добавлять новых NPC, реализовывать боевую систему, инвентарь и другие игровые элементы.

\subsection{Интеграция движка с LLM}
\subsection{YAML-описания тайлов карты и генерация игрового мира}
\subsection{Система квестов на основе графа взаимоотношений NPC}
\subsection{Хранение состояния мира в SQLite}

\section{Экспериментальная часть}
\subsection{Методика проведения экспериментов и сценарии тестирования}
\subsection{Оценка корректности вызова функций (Function calling accuracy)}
\subsection{Оценка производительности системы (время отклика, стабильность работы)}
\subsection{Пользовательское тестирование (обратная связь игроков)}

\section{Результаты}
\subsection{Анализ экспериментов по точности вызовов функций}
\subsection{Анализ производительности разработанной системы}
\subsection{Результаты пользовательского тестирования и выявленные недостатки}

\section{Заключение}
\newpage
%вот сюда пишите текст :)
\newpage
\addcontentsline{toc}{section}{Список литературы}
\bibliography{bib}

\end{document}
